\documentclass{beamer}
\usetheme{Berlin}
\usepackage{amsmath}
\usepackage{graphicx}
\setbeamercovered{dynamic}
\usepackage{lipsum}
\usepackage{hyperref}
\usepackage{colortbl}
%\usecolortheme{beaver}

\title{Assignment}
\author[]{Hamsini \\ 
Computer Science and Engineering \\ 
210010038@iitdh.ac.in}
\subtitle{CS 213, Software Systems Lab}
\date{ August 9, 2022.}
\logo{\includegraphics[width=2cm,height=1.5cm]{logo.png}}
\institute[CSE, IIT Dharwad]{Indian Institute of Technology, Dharwad}
\begin{document}
\frame[plain]{\titlepage}

\begin{frame}[label=page_2]
    Hello, this is our first frame. Here you can find info about IIT Dharwad at the below mentioned link. \url{www.iitdh.ac.in } or by clicking the following IIT Dharwad name.\\
You can find info about super heroes by clicking at the following button. 
\hyperlink{page_22}{\beamergotobutton{Super Heroes}}\\
You can find info about mathematicians by clicking the following button.
\hyperlink{page_21}{\beamergotobutton{Mathematician}}

\end{frame}

\section{Structures}

\begin{frame}{Structures}
\begin{block}{Theorem: Magic}
\alt<2>{I have done the magic. You should have trusted me earlier before I
have done the magic.}{My name is Alt Temporal. I can do magic by changing myself in different slides.}
\end{block}
\begin{alertblock}{The Truth}
\alt<2>{I was just joking, The above person actually does the magic and
also I can do it.}{No don’t trust him he doesn’t do magic.}
\end{alertblock}
\end{frame}

\begin{frame}{Structures with columns}
\begin{columns}
    \begin{column}<1->{0.4\textwidth}
      \begin{alertblock}{Alert Column}
Remain alert for the red
light.
       \end{alertblock}  
    \end{column}
    \begin{column}<3->{0.4\textwidth}
     \begin{exampleblock}{To Go}
  Green light means to go
past.  
\end{exampleblock}  
    \end{column}
\end{columns}
\begin{columns}
    \begin{column}<2->{0.4\textwidth}
     
      \begin{block}{Test your Skills}
There won’t be any blue
light at the signal.
      \end{block}
    \end{column}
    \begin{column}<4->{0.4\textwidth}
     \begin{exampleblock}{Extra Info}
  Green is also the color of
this example block.
\end{exampleblock}    
    \end{column}
\end{columns}
\end{frame}

\section{Tables}

\begin{frame}{Table Frame 1}
    \begin{table}[h]
        \centering
        \begin{tabular}{c c c}
        \hline
           Our Sir in Google Meet  & Student A &   \pause \\
              & Student B &  \pause \\
              \hline
             Student C & Student D & Student E\\
             \hline
        \end{tabular}
        \caption{ Google Meet Interface as a Table with Teacher}
        \label{Table:}
    \end{table}
    The table number shows teacher interacting with students in
Google Meet interface.
\end{frame}
\begin{frame}{Table Frame 2}
    \begin{table}[h]
        \centering
        \begin{tabular}{c}
            \hline
            Students\\
            A\hspace{2em}B\\
            \hline
        \end{tabular}
        \caption{Google Meet Interface without Teacher}
        \label{tab:my_label}
    \end{table}
    The table number shows students in Google Meet interface.
\end{frame}
\begin{frame}{Table Frame 3}
    \begin{table}[h]
        \caption{My Sample Table}
        \label{tab:my_label}
        \centering
        \begin{tabular}{c|c|c|c}
            \textbf{Value 1} & \textbf{Value 2} & \textbf{Value 3} & \textbf{Value 4}  \\
            \hline
            1 & 5 & 6 & 8\\ \pause
            2 & 9 & 11 & 13\\ \pause
            3 & 58 & 23 & 62 \\
        \end{tabular} 
    \end{table}\pause
    This is a sample table consisting of random things
\end{frame}

\begin{frame}{Table Frame 4}
    \begin{table}[h]
        \caption{Students and their Marks}
        \label{tab:my_label}
        \centering    
        \begin{tabular}
        {c c c c c}
            Sub     & A & \uncover<2->B & \uncover<3->C & \uncover<4->D \\
            Mat     & 35 & \uncover<2->{62} & \uncover<3->{93} & \uncover<4->{24} \\
            Phy     & 32 & \uncover<2->{41} & \uncover<3->{56} & \uncover<4->{96} \\
            Che     & 55 & \uncover<2->{83} & \uncover<3->{58} & \uncover<4->{92}
        \end{tabular}
    \end{table}
\end{frame}

\section{Transitions}

\begin{frame}{Transition 1}
    \lipsum[1-1]
\end{frame}
\begin{frame}{Transition 2}
    \lipsum[2-2]
\end{frame}
\begin{frame}{Transition 3}
    \lipsum[3-3]
\end{frame}
\begin{frame}{Transition 4}
    \lipsum[4-4]
\end{frame}

\section{Figures}

\begin{frame}{Great Mathematicians}
    This frame shows pictures of Great Mathematicians.
    \begin{columns}
        \hspace{2.5em}
        \begin{column}{0.3\textwidth}
            \includegraphics[scale=0.39]{aryabhatta.jpeg}
        \end{column}
        \begin{column}{0.8\textwidth}
            \includegraphics[scale=0.09]{Srinivasa.jpg}
        \end{column}
    \end{columns}   
\end{frame}

\section{Mathematics}

\begin{frame}{Mathematics}
    Mathematics is a subject that we learn in the class. This is what you may have known. But it’s something else. It’s a world of numbers.\\
    \pause
    Mathematics consists of \\
    1) Theorem or Proofs \pause\\
    2) Equations\\
    \invisible<1-2>{This is the last slide of Mathematics Intro}
    
\end{frame}

\begin{frame}{Theorems or Proofs 1}
    \begin{block}{Theorem: The Most Complicated One}
        $(a+b)^2=a^2 + b^2 + 2ab$
    \end{block}
    \begin{exampleblock}{Proof:}
    \begin{eqnarray}
            (a+b)^2 & = & (a+b)(a+b) \nonumber \\
                    & = & a^2 + ab + ba + b^2\nonumber \\
                    & = & a^2 + b^2 +2ab\nonumber
    \end{eqnarray}
    \end{exampleblock}
\end{frame}

\begin{frame}{Theorems or Proofs 2}
    \begin{block}{Theorem: The Second Most Complicated One}
        $(a-b)^2=a^2 + b^2 - 2ab$
    \end{block}
    \begin{exampleblock}{Proof:}
        \begin{eqnarray}
            (a-b)^2 & = & (a-b)(a-b) \nonumber \\
                    & = & a^2 - ab - ba + b^2\nonumber \\
                    & = & a^2 + b^2 -2ab\nonumber
    \end{eqnarray}
    \end{exampleblock}
\end{frame}

\begin{frame}{Theorem Other Way}
    \begin{block}{Theorem}
        $Multiplication \; is \; not \; Commutative\; in\; Matrices.$
    \end{block}
    \begin{block}{Proof.}
        AB is not equal to BA.
    \end{block}
\end{frame}

\begin{frame}{Equation -1}
This frame will consist of a multiline equation as follows:\\
Our Multiline Equation:\\
Area of Circle
\begin{eqnarray}
            A & = & \pi r^2  \pause\\ 
              & = &\frac{\pi d^2}{4}\nonumber 
\end{eqnarray}
\end{frame}

\begin{frame}{Equation -2}
This frame will consist of onemore multiline equation as follows:\\
Our Multiline Equation:\\
Einstein’s Equation
\begin{eqnarray}
            E & = & mc^2 \nonumber \\
    \only<2->{& = & (\sqrt{mc})^2 }\nonumber
\end{eqnarray}

\uncover<1-1>{See magic converts single line eqn to multi line eqn}\\
\only<2->{See I said no, I know magic, you should agree}
\end{frame}

\section{Our Section of Lists}

\begin{frame}{Our First List using itemize}
Hello\pause 
\begin{itemize}
    \item This is the first list item 
    \uncover<2->{\item This is the second list item}  
    \item This is the final list item which will be visible in all slides of
the frame
\end{itemize}

\end{frame}

\begin{frame}[label=page_21]{Our Second List using enumerate}
Well known famous People
\begin{enumerate}
    \item \href{https://en.wikipedia.org/wiki/Aryabhata}{Aryabhatta}\pause 
    \item \href{https://en.wikipedia.org/wiki/Albert_Einstein}{Einstein}\pause 
    \item \href{https://en.wikipedia.org/wiki/Newton}{Newton}\pause 
    \item \href{https://en.wikipedia.org/wiki/Niels_Bohr}{Bohr}\\
    You can find more info about them by clicking on their names
or by searching at Google
\end{enumerate}
\end{frame}

\begin{frame}[label=page_22]{Our Third List using Description}
\begin{description}
    \uncover<2->{\item[Iron Man] His speciality is he wears a iron man suit.}
    \uncover<3->{\item[Thor] He has a hammer.}
    \item[Vivek] He know’s how to use Beamer.
\end{description}
\alert<3>{Hereby all our lists are also completed}
\end{frame}

\begin{frame}{Our Final Frame that has Different Slides}
\color<1->{blue}{This frame is dedicated to the Beamer and people of IIT Dharwad
who are using it.}\\
\alert<1->{This is our final frame, and it says that all the given tasks are
completed by:}\\
\color<1->{blue}{To go back to the first frame after title click the following button.}\\
\hyperlink{page_2}{\beamerreturnbutton{Our First Frame}}
\pause
\end{frame}

\end{document}